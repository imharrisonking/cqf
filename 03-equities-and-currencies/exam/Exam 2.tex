\documentclass[10pt]{article}
\usepackage[utf8]{inputenc}
\usepackage[T1]{fontenc}
\usepackage{amsmath}
\usepackage{amsfonts}
\usepackage{amssymb}
\usepackage[version=4]{mhchem}
\usepackage{stmaryrd}
\usepackage{hyperref}
\hypersetup{colorlinks=true, linkcolor=blue, filecolor=magenta, urlcolor=cyan,}
\urlstyle{same}
\usepackage{bbold}

\title{Assignment for Module 3 }

\author{}
\date{}


\begin{document}
\maketitle
Summer 2023

Instructions: This is a mini project on the use of the Monte Carlo scheme to price exotic options to be completed using Python. $\mathrm{C}++$ is also allowed, but Excel/VBA is not permitted. As this is the half way point of the CQF, this assessment is designed for delegates to show independence and maturity in interpretation of a slightly open ended problem. It will test

\begin{itemize}
  \item finding and understanding the relevant lectures, Python labs and tutorials in module 3; as well as the Python primer.

  \item ability to experiment and demonstrate initiative in mathematical and numerical methods.

  \item willingness to work outside narrow instruction that are typical of maths based tests/exams.

\end{itemize}

Queries to \href{mailto:riaz.ahmad@fitchlearning.com}{riaz.ahmad@fitchlearning.com}

Task

Use the expected value of the discounted payoff under the risk-neutral density $\mathbb{Q}$

$$
V(S, t)=e^{-r(T-t)} \mathbb{E}^{\mathbb{Q}}\left[\text { Payoff }\left(S_{T}\right)\right]
$$

for the appropriate form of payoff, to consider Asian and lookback options.

Use the Euler-Maruyama (only) scheme for initially simulating the underlying stock price. As an initial example you may use the following set of sample data

$$
\begin{aligned}
\text { Today's stock price } S_{0} & =100 \\
\text { Strike } E & =100 \\
\text { Time to expiry }(T-t) & =1 \text { year } \\
\text { volatility } \sigma & =20 \% \\
\text { constant risk-free interest rate } r & =5 \%
\end{aligned}
$$

Then vary the data to see the affect on the option price. Your completed assignment should centre on a report to include:

$\begin{array}{ll}\text { Outline of the finance problem and numerical procedure used } & \text { Mark } \\ \text { Results - appropriate tables and comparisons } & 35 \% \\ \text { Any interesting observations and problems encountered } & 25 \% \\ \text { Conclusion } & 15 \% \\ \text { References } & 5 \%\end{array}$

\begin{itemize}
  \item Outline of the finance problem and numerical procedure used.

  \item Results - appropriate tables and comparisons.

  \item Any interesting observations and problems encountered.

  \item Conclusion and references

\end{itemize}

For a Python Jupyter Notebook, a detailed notebook will become the complete report (writeup, code, results). Score key

60-65 Pass

66-70 Good

71-79 Very Good

80-89 Excellent

90-95 Outstanding

96+ Exceptional

Note: An assessment of this form differs from mathematical exercises that can attract full marks. The key above is provided for this reason.


\end{document}