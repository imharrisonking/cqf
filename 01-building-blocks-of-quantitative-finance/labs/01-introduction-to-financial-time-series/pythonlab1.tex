\documentclass[10pt]{article}
\usepackage[utf8]{inputenc}
\usepackage[T1]{fontenc}
\usepackage{graphicx}
\usepackage[export]{adjustbox}
\graphicspath{ {./images/} }
\usepackage{amsmath}
\usepackage{amsfonts}
\usepackage{amssymb}
\usepackage[version=4]{mhchem}
\usepackage{stmaryrd}
\usepackage{hyperref}
\hypersetup{colorlinks=true, linkcolor=blue, filecolor=magenta, urlcolor=cyan,}
\urlstyle{same}

\title{Introduction to Financial Timeseries }


\author{Kannan Singaravelu, CQF}
\date{}


\begin{document}
\maketitle
June 2023

\section*{Financial Data Preprocessing}
A time series is a series of data points indexed in time order. Financial Data such as equity, commodity, and forex price series observed at equally spaced points in time are an example of such a series. It is a sequence of data points observed at regular time intervals and depending on the frequency of observations, a time series may typically be in ticks, seconds, minutes, hourly, daily, weekly, monthly, quarterly and annual.

The first step towards any data analysis would be to parse the raw data that involves extracting the data from the source and then cleaning and filling the missing data if any. While data comes in many forms, Python makes it easy to read time-series data using useful packages.

In this session, we will retrieve and store both end-of-day and intraday data using some of the popular python packages. These libraries aim to keep the API simple and make it easier to access historical data. Further, we will see how to read data from traditional data sources stored locally.

\subsection*{Load Libraries}
We'll now import the required libraries that we'll use in this example. Refer requirements.txt for list of packages.

\begin{center}
\includegraphics[max width=\textwidth]{2023_06_30_e9266d5db76d1dc63d44g-1}
\end{center}

\subsection*{Docstring or Signature}
Getting information on function attributes and outputs. [ ]: \#help(yf.download)

\# yf. download?

\subsection*{Data Retrieval}
Retrieving EOD, Intraday, Options data

\subsubsection*{Retrieving end-of-day data for single security}
We'll retrieve historical data from yahoo finance using yfinance library

\section*{Example 1}
\begin{center}
\includegraphics[max width=\textwidth]{2023_06_30_e9266d5db76d1dc63d44g-2(1)}
\end{center}

\section*{Example 2}
\begin{center}
\includegraphics[max width=\textwidth]{2023_06_30_e9266d5db76d1dc63d44g-2(2)}
\end{center}

\section*{Example 3}
[ ]: \#Fetch data for year to date (YTD)

$\mathrm{df} 3$ = yf.download('SPY', period='ytd', progress=False)

\# Display the last five rows of the dataframe to check the results. $\operatorname{df} 3 \cdot \operatorname{tail}()$

\subsubsection*{Retrieving data for multiple securities}
We'll retrieve historical price data of five Nasdaq-listed stocks from yahoo finance.

\section*{Example 4}
\begin{center}
\includegraphics[max width=\textwidth]{2023_06_30_e9266d5db76d1dc63d44g-2}
\end{center}

\subsubsection*{Retrieving multiple fields for multiple securities}
We'll now retrieve multiple fields from yahoo finance.

\section*{Example 5}
\begin{center}
\includegraphics[max width=\textwidth]{2023_06_30_e9266d5db76d1dc63d44g-3}
\end{center}

[]$:$ ohlcv

[ ]: \# Display NVDA stock data $\operatorname{ohlcv}\left[' G S^{\prime}\right] \cdot \operatorname{head}()$

[ ]: \# Display GS adjusted close data ohlcv['GS'] ['Adj Close']

\subsubsection*{Retrieving intraday data}
We'll now retrieve intraday data from yahoo finance.

\section*{Example 6}
[ ]: \#Retrieve intraday data for last five days

df6 = yf.download('SPY', period='5d', interval='1m', progress=False)

\# Display dataframe

$\mathrm{df} 6$

\subsubsection*{Retrieving option chain}
We'll now retrieve option chain for SPY for March 2022 expiration from yahoo finance and filter the output to display the first seven columns.

\section*{Example 7}
[]$:$ spy $=$ yf.Ticker $('$ SPY')

[ ] : spy.option\_chain

[ ]: \#Get SPY option chain for Sept 30th expiration

\# \href{https://finance.yahoo}{https://finance.yahoo}. com/quote/SPY/options?date $=1680220800$

\# spy $=$ yf.Ticker('SPY')

options = spy.option\_chain('2023-06-30')

options

[ ]: \# Filter calls for strike above 390

df7 = options.calls[options.calls['strike']>390]

\# Check the filtered output

$\operatorname{df} 7 . i \operatorname{loc}[:,: 7]$

\subsubsection*{Retrieving Hypertext Markup Language (HTML)}
We'll now retrieve India's benchmark index NIFTY50 Index data from Wikipedia.

\section*{Example 8}
\begin{center}
\includegraphics[max width=\textwidth]{2023_06_30_e9266d5db76d1dc63d44g-4}
\end{center}

\subsection*{Data Storage}
Export files and store it in a local drive

\subsubsection*{Storing OHLCV data in Excel File}
\begin{center}
\includegraphics[max width=\textwidth]{2023_06_30_e9266d5db76d1dc63d44g-4(2)}
\end{center}

\subsubsection*{Storing OHLCV data in a CSV File}
\begin{center}
\includegraphics[max width=\textwidth]{2023_06_30_e9266d5db76d1dc63d44g-4(1)}
\end{center}

\subsection*{Data Loading}
Importing files stored in the local drive

\subsubsection*{Reading Microsfot Excel File}
We'll now read the Excel file stored locally using Pandas

\begin{center}
\includegraphics[max width=\textwidth]{2023_06_30_e9266d5db76d1dc63d44g-4(3)}
\end{center}

\subsubsection*{Reading CSV File}
We'll now read the csv file stored locally using Pandas

\begin{center}
\includegraphics[max width=\textwidth]{2023_06_30_e9266d5db76d1dc63d44g-5}
\end{center}

\subsection*{Interactive Visualization of Time Series}
We use cufflinks for interactive visualization. It is one of the most feature rich third-party wrapper around Plotly by Santos Jorge. It binds the power of plotly with the flexibility of pandas for easy plotting.

When you import cufflinks library, all pandas data frames and series objects have a new method .iplot() attached to them which is similar to pandas' built-in .plot() method.

\subsubsection*{Plotting Line Chart}
Next, we'll plot the time series data in the line format.

[ ]: df3['Close'][-30:].iplot(kind='line', title='Line Chart')

\subsubsection*{Plotting OHLC Data}
Next, we'll plot the time series data in ohlc format.

[ ]: df3[-30:].iplot(kind='ohlc', title='Bar Chart')

\subsubsection*{Plotting Candlestick}
Next, we'll plot an interactive candlestick chart.

[ ]: df3[-30:].iplot(kind='candle', title='Candle Chart')

\subsubsection*{Plotting Selected Stocks}
Next, we'll compare the GS \& HD data that we fetched from Yahoo Finance.

[]$:$ \# Use secondary axis

$\operatorname{df} 4[[' G S '$, 'HD'] ].iplot(secondary\_y='HD')

\subsubsection*{Plotting using Subplots}
[]$:$ \# Use subplots

$\operatorname{df} 4[[$ GS', 'HD'] ].iplot(subplots=True)

\subsubsection*{Normalized Plot}
[ ] : df4.normalize().iplot()

\subsubsection*{Visualising Return Series}
We'll now plot historical daily log normal return series using just one line of code.

\begin{center}
\includegraphics[max width=\textwidth]{2023_06_30_e9266d5db76d1dc63d44g-6(1)}
\end{center}

Log Normal Distribution A normal distribution is the most common and widely used distribution in statistics. It is popularly referred as a "bell curve" or "Gaussian curve". Financial time series though random in short term, follows a log normal distribution on a longer time frame.

Now that we have derived the daily log returns, we will plot this return distribution and check whether the stock returns follows log normality.

\begin{center}
\includegraphics[max width=\textwidth]{2023_06_30_e9266d5db76d1dc63d44g-6(2)}
\end{center}

\section*{Plotting Annual Returns}
[ ]: \#Plot Mean Annual Returns

(daily\_returns $\cdot \operatorname{mean}() * 252 * 100)$ ) iplot (kind=' bar')

\section*{Plotting Annualized Volatility}
[ ]: \# Plot Mean Annualized Volatility

(daily\_returns.std()*np.sqrt (252)*100) . iplot (kind=' bar')

Plotting Correlation Correlation defines the similarity between two random variables. As an example we will check correlation between our Nasdaq listed stocks.

\begin{center}
\includegraphics[max width=\textwidth]{2023_06_30_e9266d5db76d1dc63d44g-6}
\end{center}

\subsection*{References}
\begin{itemize}
  \item YFinance Documentation

  \item Cufflinks Documentation

  \item Numpy Documentation

  \item Pandas Documentation

  \item Python Resources

\end{itemize}

June 2023, Certificate in Quantitative Finance.


\end{document}